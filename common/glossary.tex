\newacronym{abus}{ABUS SC}{ABUS Security Center GmbH \& Co. KG}
\newacronym{portal}{Portal}{ABUS Cloud-Portal}

\newacronym{ts}{TS}{\Gls{ts-gls}}
\newacronym{pm}{PM}{\Gls{pm-gls}}

\newacronym{di}{DI}{Dependency Injection}

\newglossaryentry{collection}
{
    name=collection,
    description={
        Im Kontext der MongoDB ist eine Collection ein Container für Dokumente mit (meist) ähnlicher Struktur,
        ähnlich wie eine Tabelle in einer relationalen Datenbank}
}

\newglossaryentry{mongodb}
{
    name=MongoDB,
    description={
        MongoDB ist eine dokumentenorientierte NoSQL-Datenbank. Anstatt Daten in Tabellen zu speichern,
        wie es bei relationalen Datenbanken der Fall ist, speichert MongoDB Daten in
        flexiblen BSON \textit{(Binary JSON)} Dokumenten}
}

\newglossaryentry{ts-gls}
{
    name=Technischer Support,
    description={
        Der \acrfull{ts} ist eine Mitarbeiterrolle bei \acrshort{abus}, die Kunden bei technischen Problemen mit
        ihren Geräten unterstützt. Ein \acrshort{ts} hat Zugriff auf das \acrshort{portal} und kann dort
        Geräte-Details einsehen, Fernwartungsaktionen durchführen etc. Dafür benötigt er aber von dem jeweiligen
        \Gls{facherrichter} oder \Gls{endnutzer} eine Freigabe}
}

\newglossaryentry{pm-gls}
{
    name=Produktmanager,
    description={
        Der \acrfull{pm} ist eine Mitarbeiterrolle bei \acrshort{abus}, die für eine Produktlinie
        (z.B. Secoris) verantwortlich ist. Ein \acrshort{pm} kann ungenutzte Geräte im \acrshort{portal}
        zurücksetzen/löschen und die Firmware-Versionen verwalten. Außerdem ist er die nächste
        Anspruchsperson für den \acrshort{ts} bei Problemen, die dieser nicht lösen kann}
}

\newglossaryentry{facherrichter}
{
    name=Facherrichter,
    description={
        Ein Facherrichter ist ein Händler oder Vertriebspartner, der die Produkte von \acrshort{abus}
        an Endkunden verkauft. Sie sind Kunden von \acrshort{abus}. \\
        Ein Facherrichter installiert und ggf. wartet Geräte von
        \acrshort{abus} bei einem Endkunden, z.B. einem Lebensmittelladen. Bei Problemen mit einem Gerät
        sind die Facherrichter der erste Ansprechpartner. Sie können alle Gerätedetails im Cloud-Portal
        sehen und das Gerät aus der Ferne warten. Falls sie nicht weiterhelfen können, erreicht die
        Anfrage den TS}
}

\newglossaryentry{endnutzer}
{
    name=Endnutzer,
    description={
        Die Endnutzer können sowohl Privat- als auch Gewerbekunden sein.
        Sie sind Kunden von \acrshort{abus}. Privatkunden dürfen ein
        Gerät selbst installieren und verwalten. Gewerbekunden dagegen müssen sich aus
        Versicherungsschutzgründen an einen qualifizierten Facherrichter wenden}
}