\newacronym{abus}{ABUS SC}{ABUS Security Center GmbH \& Co. KG}
\newacronym{portal}{Portal}{ABUS Cloud-Portal}

\newacronym{ts}{TS}{\Gls{ts-gls}}
\newacronym{pm}{PM}{\Gls{pm-gls}}

\newacronym{di}{DI}{Dependency Injection}

\newglossaryentry{mongodb}
{
    name=MongoDB,
    sort=tech1,
    description={
        MongoDB ist eine dokumentenorientierte NoSQL-Datenbank. Anstatt Daten in Tabellenspalten zu speichern,
        wie es bei relationalen Datenbanken der Fall ist, speichert MongoDB Daten in
        flexiblen BSON \textit{(Binary JSON)} Dokumenten, die in \Gls{collection}s organisiert sind
        \cite{mongodb,mongodb-collections}}
}

\newglossaryentry{collection}
{
    name=Collection,
    sort=tech2,
    description={
        Im Kontext der MongoDB ist eine Collection ein Container für Dokumente mit (meist) ähnlicher Struktur,
        ähnlich wie eine Tabelle in einer relationalen Datenbank
        \cite{mongodb-collections}}
}

\newglossaryentry{rwd}
{
    name=Responsive Web Design,
    sort=tech3,
    description={
        Laut MDN\cite{rwd} ist Responsive Web Design (RWD) \textit{"ein Ansatz im Webdesign, um sicherzustellen,
        dass Webseiten auf allen Bildschirmgrößen und -auflösungen gut dargestellt werden und
        gleichzeitig eine gute Benutzerfreundlichkeit gewährleisten"}}
}

\newglossaryentry{routing}
{
    name=Routing,
    sort=tech4,
    description={
        Routing im Kontext von \textit{Single-Page Applications} (SPAs) bezieht sich auf den Mechanismus,
        der es ermöglicht, den Seiteninhalt anhand der URL dynamisch zu ändern, ohne die gesamte Seite neu zu laden.
        Wie das in Angular funktioniert, wird in \cite{routing} beschrieben}
}

\newglossaryentry{repository}
{
    name=Repository,
    sort=tech5,
    description={
        Das Repository-Muster ist ein Entwurfsmuster im \textit{Domain-Driven Design} (DDD), das eine Abstraktionsschicht
        zwischen der Domänenschicht und der Datenzugriffsschicht bereitstellt. Es kapselt die Logik zum Abrufen,
        Speichern und Verwalten von Daten. Repositories werden meist als Interfaces in der Domänenschicht definiert
        und in der Datenzugriffsschicht implementiert \cite{repository}}
}

\newglossaryentry{controller}
{
    name=Controller,
    sort=tech6,
    description={
        Ein Controller ist im Kontext von ASP.NET Web API eine Klasse, die API-Endpunkte definiert und
        HTTP-Anfragen verarbeitet. Eine Controller-Methode entspricht einem Endpunkt und wird durch
        Attribute wie \texttt{[HttpGet]}, \texttt{[HttpPost]} usw. gekennzeichnet \cite{controller}}
}

\newglossaryentry{ts-gls}
{
    name=Technischer Support,
    sort=role3,
    description={
        Der \acrfull{ts} ist eine Mitarbeiterrolle bei \acrshort{abus}, die Kunden bei technischen Problemen mit
        ihren Geräten unterstützt. Ein \acrshort{ts} hat Zugriff auf das \acrshort{portal} und kann dort
        Geräte-Details einsehen, Fernwartungsaktionen durchführen etc. Dafür benötigt er aber von dem jeweiligen
        \Gls{facherrichter} oder \Gls{endnutzer} eine Freigabe}
}

\newglossaryentry{pm-gls}
{
    name=Produktmanager,
    sort=role4,
    description={
        Der \acrfull{pm} ist eine Mitarbeiterrolle bei \acrshort{abus}, die für eine Produktlinie
        (z.B. Secoris) verantwortlich ist. Ein \acrshort{pm} kann ungenutzte Geräte im \acrshort{portal}
        zurücksetzen/löschen und die Firmware-Versionen verwalten. Außerdem ist er die nächste
        Anspruchsperson für den \acrshort{ts} bei Problemen, die dieser nicht lösen kann}
}

\newglossaryentry{facherrichter}
{
    name=Facherrichter,
    sort=role2,
    description={
        Ein Facherrichter ist ein Händler oder Vertriebspartner, der die Produkte von \acrshort{abus}
        an Endkunden verkauft. Sie sind Kunden von \acrshort{abus}. \\
        Ein Facherrichter installiert und ggf. wartet Geräte von
        \acrshort{abus} bei einem Endkunden, z.B. einem Lebensmittelladen. Bei Problemen mit einem Gerät
        sind die Facherrichter der erste Ansprechpartner. Sie können alle Gerätedetails im Cloud-Portal
        sehen und das Gerät aus der Ferne warten. Falls sie nicht weiterhelfen können, erreicht die
        Anfrage den TS}
}

\newglossaryentry{endnutzer}
{
    name=Endnutzer,
    sort=role1,
    description={
        Die Endnutzer können sowohl Privat- als auch Gewerbekunden sein.
        Sie sind Kunden von \acrshort{abus}. Privatkunden dürfen ein
        Gerät selbst installieren und verwalten. Gewerbekunden dagegen müssen sich aus
        Versicherungsschutzgründen an einen qualifizierten Facherrichter wenden}
}
