\documentclass[11pt]{article}
\usepackage[german]{babel}
\usepackage[german=quotes]{csquotes}
\usepackage[a4paper, top=2cm, bottom=2cm, left=2cm, right=1.5cm]{geometry}
\usepackage{amsmath, graphicx, listings, parskip, float, lastpage, fancyhdr, booktabs}
\usepackage[colorlinks=true, allcolors=blue]{hyperref}
\usepackage{microtype} % Prevents text overflow with micro-typographic adjustments
\sloppy % Allows more flexible line breaking
\usepackage[acronym]{glossaries}
\usepackage{lipsum}
\usepackage{colortbl}

\definecolor{lightgray}{gray}{0.9}
\definecolor{lightgreen}{cmyk}{0.43, 0, 0.79, 0}

\newcommand{\authorName}{Ilia Bozhkov}
\newcommand{\authorStreet}{Brendelstraße 6}
\newcommand{\authorLocation}{90453 Nürnberg}

\newcommand{\companyName}{conplement AG}
\newcommand{\companyStreet}{Südwestpark 92}
\newcommand{\companyLocation}{90449 Nürnberg}

\newcommand{\documentPurpose}{Dokumentation zur betrieblichen Projektarbeit}
\newcommand{\examDate}{Winter 2025}
\newcommand{\filingDate}{\today}

\newcommand{\projectSubTitle}{Erweiterung des bestehenden ABUS Cloud-Portals um eine
Auswertungsansicht für Gerätenachrichten, damit der technische Support
Fehler in unterstützten Zutritts- und Alarmsystemen schneller erkennen kann.
}
\newcommand{\projectTitle}{Auswertungsansicht für Gerätenachrichten im ABUS Cloud-Portal}

\newcommand{\qualifiedJobTitle}{Fachinformatiker:in für Anwendungsentwicklung}

\makeglossaries
\newacronym{abus}{ABUS SC}{ABUS Security Center GmbH \& Co. KG}
\newacronym{portal}{Portal}{ABUS Cloud-Portal}

\newacronym{ts}{TS}{\Gls{ts-gls}}
\newacronym{pm}{PM}{\Gls{pm-gls}}

\newacronym{di}{DI}{Dependency Injection}

\newglossaryentry{mongodb}
{
    name=MongoDB,
    sort=tech1,
    description={
        MongoDB ist eine dokumentenorientierte NoSQL-Datenbank. Anstatt Daten in Tabellenspalten zu speichern,
        wie es bei relationalen Datenbanken der Fall ist, speichert MongoDB Daten in
        flexiblen BSON \textit{(Binary JSON)} Dokumenten, die in \Gls{collection}s organisiert sind
        \cite{mongodb,mongodb-collections}}
}

\newglossaryentry{collection}
{
    name=Collection,
    sort=tech2,
    description={
        Im Kontext der MongoDB ist eine Collection ein Container für Dokumente mit (meist) ähnlicher Struktur,
        ähnlich wie eine Tabelle in einer relationalen Datenbank
        \cite{mongodb-collections}}
}

\newglossaryentry{rwd}
{
    name=Responsive Web Design,
    sort=tech3,
    description={
        Laut MDN\cite{rwd} ist Responsive Web Design (RWD) \textit{"ein Ansatz im Webdesign, um sicherzustellen,
        dass Webseiten auf allen Bildschirmgrößen und -auflösungen gut dargestellt werden und
        gleichzeitig eine gute Benutzerfreundlichkeit gewährleisten"}}
}

\newglossaryentry{routing}
{
    name=Routing,
    sort=tech4,
    description={
        Routing im Kontext von \textit{Single-Page Applications} (SPAs) bezieht sich auf den Mechanismus,
        der es ermöglicht, den Seiteninhalt anhand der URL dynamisch zu ändern, ohne die gesamte Seite neu zu laden.
        Wie das in Angular funktioniert, wird in \cite{routing} beschrieben}
}

\newglossaryentry{repository}
{
    name=Repository,
    sort=tech5,
    description={
        Das Repository-Muster ist ein Entwurfsmuster im \textit{Domain-Driven Design} (DDD), das eine Abstraktionsschicht
        zwischen der Domänenschicht und der Datenzugriffsschicht bereitstellt. Es kapselt die Logik zum Abrufen,
        Speichern und Verwalten von Daten. Repositories werden meist als Interfaces in der Domänenschicht definiert
        und in der Datenzugriffsschicht implementiert \cite{repository}}
}

\newglossaryentry{controller}
{
    name=Controller,
    sort=tech6,
    description={
        Ein Controller ist im Kontext von ASP.NET Web API eine Klasse, die API-Endpunkte definiert und
        HTTP-Anfragen verarbeitet. Eine Controller-Methode entspricht einem Endpunkt und wird durch
        Attribute wie \texttt{[HttpGet]}, \texttt{[HttpPost]} usw. gekennzeichnet \cite{controller}}
}

\newglossaryentry{ts-gls}
{
    name=Technischer Support,
    sort=role3,
    description={
        Der \acrfull{ts} ist eine Mitarbeiterrolle bei \acrshort{abus}, die Kunden bei technischen Problemen mit
        ihren Geräten unterstützt. Ein \acrshort{ts} hat Zugriff auf das \acrshort{portal} und kann dort
        Geräte-Details einsehen, Fernwartungsaktionen durchführen etc. Dafür benötigt er aber von dem jeweiligen
        \Gls{facherrichter} oder \Gls{endnutzer} eine Freigabe}
}

\newglossaryentry{pm-gls}
{
    name=Produktmanager,
    sort=role4,
    description={
        Der \acrfull{pm} ist eine Mitarbeiterrolle bei \acrshort{abus}, die für eine Produktlinie
        (z.B. Secoris) verantwortlich ist. Ein \acrshort{pm} kann ungenutzte Geräte im \acrshort{portal}
        zurücksetzen/löschen und die Firmware-Versionen verwalten. Außerdem ist er die nächste
        Anspruchsperson für den \acrshort{ts} bei Problemen, die dieser nicht lösen kann}
}

\newglossaryentry{facherrichter}
{
    name=Facherrichter,
    sort=role2,
    description={
        Ein Facherrichter ist ein Händler oder Vertriebspartner, der die Produkte von \acrshort{abus}
        an Endkunden verkauft. Sie sind Kunden von \acrshort{abus}. \\
        Ein Facherrichter installiert und ggf. wartet Geräte von
        \acrshort{abus} bei einem Endkunden, z.B. einem Lebensmittelladen. Bei Problemen mit einem Gerät
        sind die Facherrichter der erste Ansprechpartner. Sie können alle Gerätedetails im Cloud-Portal
        sehen und das Gerät aus der Ferne warten. Falls sie nicht weiterhelfen können, erreicht die
        Anfrage den TS}
}

\newglossaryentry{endnutzer}
{
    name=Endnutzer,
    sort=role1,
    description={
        Die Endnutzer können sowohl Privat- als auch Gewerbekunden sein.
        Sie sind Kunden von \acrshort{abus}. Privatkunden dürfen ein
        Gerät selbst installieren und verwalten. Gewerbekunden dagegen müssen sich aus
        Versicherungsschutzgründen an einen qualifizierten Facherrichter wenden}
}


% The command \setlength{\parskip}{...} sets the length of the vertical space between paragraphs. By default, LaTeX adds vertical space between paragraphs to help distinguish them visually. This space is known as the "paragraph skip" or "parskip," for short.
\setlength{\parskip}{6pt}
\setlength{\headheight}{20pt}


\begin{document}


% The \pagenumbering{gobble} command removes page numbering from the title page, the abstract and the table of contents. 
% It is suspended by the later use of \pagenumbering{arabic}.
\pagenumbering{gobble}
\begin{titlepage}

    \centering
    \vspace*{\fill}
    
    {\normalsize  Abschlussprüfung \examDate\par}
    \vspace{0.5cm}
    {\normalsize \qualifiedJobTitle\par}
    \vspace*{2cm}
    
    {\large \documentPurpose\par}
    \vspace*{1cm}
    
    {\LARGE \bfseries \projectTitle\par}
    \vspace{0.5cm}
    {\large \bfseries \projectSubTitle\par}
    \vspace*{2cm}
    
    {\normalsize \bfseries Prüfungsbewerber:in\par}
    \vspace{0.2cm}
    {\normalsize \authorName\par}
    {\normalsize \authorStreet\par}
    {\normalsize \authorLocation\par}
    \vspace*{2cm}        

    {\normalsize \bfseries Ausbildungsbetrieb:\par}
    \vspace{0.2cm}
    {\normalsize \companyName\par}
    {\normalsize \companyStreet\par}
    {\normalsize \companyLocation\par}
    \vspace*{3cm}
    
    {\normalsize \bfseries Abgabedatum:\par}
    \vspace{0.2cm}
    {\normalsize  \filingDate\par}
    
    \vspace*{\fill}
    \vspace*{\fill}

\end{titlepage}

\pagebreak


% This command changes the color of links only for the table of contents.
{
    \hypersetup{linkcolor=black}
    \tableofcontents
}
\pagebreak

\printglossary[nonumberlist, title=Glossar]\label{sec:glossary}
\printglossary[type=\acronymtype, nonumberlist, title=Abkürzungsverzeichnis]
\pagebreak

% The \pagenumbering{arabic} command is used on the first actual page (i.e., after the table of contents) to start page numbering with Arabic numerals.
% \pagenumbering{arabic} suspends the previous \pagenumbering{gobble} command.
\pagenumbering{arabic}


% This command applies some styling to the header and footer.
\pagestyle{fancy}


% Both commands clear the default header and footer content, respectively.
\fancyhead{}
\fancyfoot{}


% This command displays the project name on the right side of the header
% \projectTitle is defined in the metadata.tex file
\rhead{\projectTitle}


% This command displays the author's name on the left side of the footer
% \authorName is defined in the metadata.tex file
\lfoot{\authorName}


% \rfoot: This sets the footer's content on the page's right side.
% \thepage: This is a LaTeX command that expands to the current page number.
% \pageref*{LastPage}: This command extends to the page number of the last page in the document. The * after \pageref suppresses the hyperlink usually created if you used just \pageref{LastPage}.
\rfoot{\thepage\ von \pageref*{LastPage}}


\section{Einleitung}

    \subsection{Projektbeschreibung}

    todo

    \subsection{Projektziel}

    todo

    \subsection{Projektumfeld}

    todo

    \subsection{Projektschnittstellen}

    todo

    \subsection{Projektabgrenzung}

    todo

\section{Analysephase}

    \subsection{Wirtschaftlichkeitsanalyse} \label{sec:profitability-analysis}

        Für die conplement AG lohnt sich das Projekt schon deswegen, weil im Vertrag
        mit der \acrshort{abus} zusätzlich zu den Projektkosten auch ein Gewinnaufschlag und
        ein Risikozuschlag vereinbart wurde (die genaue Höhe ist vertraulich). Somit sind die
        Projektkosten gedeckt und es wird zusätzlich ein Gewinn erzielt.

        \subsubsection{Projektkosten}

        Die Projektkosten setzen sich aus den Personalkosten der beteiligten Mitarbeiter
        der conplement AG zusammen. Alle verwendeten Ressourcen sind bereits im Stundensatz
        der Mitarbeiter einkalkuliert, sodass keine zusätzlichen Kosten anfallen.

        Für einen Azubi (mich) ist ein Stundensatz von 35€ veranschlagt, während für
        einen erfahrenen Softwareentwickler ein Stundensatz von 104€ gilt. Das Projekt
        umfasst insgesamt 80 Stunden, die sich wie folgt aufteilen:

        \begin{itemize}
            \item 80 Stunden Arbeit von mir als Azubi
            \item 5 Stunden Arbeit von einem erfahrenen Softwareentwickler zur
            Unterstützung bei der Planung und Code-Reviews
        \end{itemize}

        Daraus ergibt sich die folgende Berechnung der Projektkosten:

        \begin{table}[H]
            \centering
            \begin{tabular}{l r r r}
                \toprule
                \textbf{Rolle} & \textbf{Stundensatz} & \textbf{Stunden} & \textbf{Gesamtkosten} \\
                \midrule
                Azubi & 35€ & 80 & 2.800€ \\
                Erfahrener Entwickler & 104€ & 5 & 520€ \\
                \midrule
                \textbf{Gesamt} & & & \textbf{3.320€} \\
                \bottomrule
            \end{tabular}
            \caption{Projektkostenübersicht}
        \end{table}

        \subsubsection{Amortisationsrechnung aus Kundensicht}

        Genaue Zahlen zur Amortisation des Projekts sind schwer zu ermitteln,
        da das Projekt keinen direkten Umsatz für \acrshort{abus} generiert.
        Stattdessen spart das Projekt indirekt Kosten, indem es die Effizienz des
        \acrshort{ts} verbessert.

        Basierend auf Gesprächen mit der \acrshort{abus} und Annahmen über die
        Arbeitsweise des \acrshort{ts} lassen sich folgende Schätzungen treffen:

        \begin{itemize}
            % \item Ein \acrshort{ts} arbeitet 35 Stunden pro Woche. - nicht relevant
            \item Der Stundensatz eines \acrshort{ts} beträgt ca. 90€ (angenommen).
            \item Durch das neue Feature wird 30 Minuten weniger für eine Support-Anfrage
            benötigt (angenommen).
            \item Bei ungefähr 10\% aller Geräte wird pro Jahr eine Support-Anfrage gestellt,
            die durch das neue Feature vereinfacht wird (angenommen).
            \item Es sind insgesamt 6.572 Geräte im \acrshort{portal} registriert (Stand: Dezember 2025).
        \end{itemize}

        Daraus ergeben sich folgende Berechnungen:

        \begin{itemize}
            \item Anzahl der Support-Anfragen pro Jahr:
            \[
                6.572 \text{ Geräte} \times 10\% = 657{,}2 \approx 657 \text{ Anfragen}
            \]
            \item Zeitersparnis pro Jahr durch das neue Feature:
            \[
                657 \text{ Anfragen} \times 0{,}5 \text{ Stunden} = 328{,}5 \text{ Stunden}
            \]
            \item Kosteneinsparung pro Jahr:
            \[
                328,5 \text{ Stunden} \times 90\text{€}/\text{Stunde} = 29.565\text{€}
            \]
            \item Kosteneinsparung pro Monat:
            \[
                29.565\text{€} / 12 = 2.463{,}75\text{€}
            \]
        \end{itemize}

        Also wird sich das Projekt für die \acrshort{abus} bereits nach etwa
        \[
            3.320\text{€} / 2.463{,}75\text{€} \approx 1{,}35 \text{ Monate} \approx 2 \text{ Monate}
        \]
        amortisieren, wenn die Annahmen zutreffen.
        
        Aber auch bei Abweichungen von diesen Annahmen ist es sehr wahrscheinlich,
        dass sich das Projekt innerhalb weniger Monate amortisiert. Z.B. wenn nur für 5\%
        der Geräte pro Jahr eine Support-Anfrage gestellt wird, bei der das neue Feature nur 15
        Minuten spart und der Stundensatz eines \acrshort{ts} nur 75€ beträgt, würde sich
        das Projekt immer noch innerhalb von etwa einem halben Jahr amortisieren.
        
        \subsubsection{Nicht-monetäre Vorteile für den Kunden}

        Es gibt auch mehrere nicht-monetäre bzw. schwer berechenbare Vorteile für die \acrshort{abus}
        durch das neue Feature:

        \begin{itemize}
            \item Weniger Freigabeanfragen und somit mehr Zufriedenheit bei \Gls{endnutzer} und \Gls{facherrichter}
            \item Verringerte Arbeitslast der \acrshort{ts} und \acrshort{pm}
            \item Weniger Support-Anfragen, die an das Entwicklungsteam eskalieren
            \item Neue Sichtweise auf die Gerätedaten und Erkenntnisse, die vorher nicht möglich waren
        \end{itemize}

    \subsection{Ist-Analyse}

    Das vorhandene System besteht aus einem Backend, einem Frontend und einer \Gls{mongodb}-Datenbank.
    Das Backend ist mit \textit{ASP.NET} implementiert und stellt eine \textit{REST-API} bereit, die vom
    \textit{Angular}-Frontend genutzt wird. Das Backend hat außerdem viele Schnittstellen zu weiteren
    internen und externen Systemen, wie z.B. einen \textit{Service Bus}, ein \textit{IoT Hub}, Systeme von
    \acrshort{abus} usw. Das Frontend wird von einem \textit{Node.js}-Server an die Clients ausgeliefert, der
    gleichzeitig auch als ein \textit{Proxy} für die API-Anfragen fungiert.
    Eine abstrakte Übersicht der Architektur ist in der Abbildung \ref{fig:system-architecture} dargestellt.

    Die \Gls{mongodb}-Datenbank speichert alle Gerätedaten, darunter auch die
    Gerätenachrichten, die für die Auswertung benötigt werden. Sie sind in einer sog.
    \Gls{collection} gespeichert, die den Namen \texttt{common-messages} trägt. Jedes Dokument in
    dieser \Gls{collection} repräsentiert eine einzelne Gerätenachricht und hat folgendes Format
    (nur relevante Felder dargestellt):

    \begin{verbatim}
    {
        deviceId: Guid,             // ID des Geräts, das die Nachricht gesendet hat
        messageType: "string",      // Typ der Nachricht (z.B. "connectEvent")
        deviceType: "string",       // Typ des Geräts (z.B. "Secoris")
        enqueueTimeUtc: Date,       // Eingangszeit der Nachricht in UTC
        rawPayload: object          // Rohdaten der Nachricht
        ...                         // weitere Felder
    }
    \end{verbatim}

    Grundsätzlich soll als Nachrichtenart der \texttt{messageType} ausgewertet werden.
    Allerdings ist dieser Typ nicht immer aussagekräftig genug, da manche Gerätetypen
    auch viel detailliertere Typen in den \texttt{rawPayload}-Daten haben. Daher
    muss die genaue Auswertung je nach Gerätetyp unterschiedlich implementiert werden.
\section{Konzeptionsphase} \label{sec:conceptualize}

    \subsection{Namensgebung}

    Das zu implementierende Feature wird im Code als \textit{Device Analysis} bzw. \textit{Message Statistics} bezeichnet.
    Device Analysis ist der Oberbegriff für alle Arten von Auswertungen, die in der Zukunft möglicherweise hinzugefügt werden.
    Message Statistics ist die konkrete Auswertung der Gerätenachrichten, die in diesem Projekt implementiert wird.
    Die beiden Begriffe werden je nach Kontext synonym verwendet.

    \subsection{Backend-Architektur}

    Das Backend ist in ASP.NET Core als ein \textit{modularer Monolith} aufgebaut. Es besteht aus dem Common-Modul und mehreren Device-Modulen, die keine
    Abhängigkeiten untereinander haben dürfen. Eine visuelle Darstellung der Architektur ist in der folgenden Abbildung dargestellt:

    \begin{figure}[H]
        \centering
        \includegraphics[width=0.9\textwidth]{../figures/backend-overview.png}
        \caption{
            Allgemeine Übersicht der Backend-Architektur
        }
    \end{figure}

    Die Module haben folgende Eigenschaften:
    \begin{itemize}
        \item Das \texttt{Common-Modul} stellt geräteübergreifende Features und Funktionalitäten bereit, wie z.B. Authentifizierung, Profilverwaltung
        und Gerätesuche.
        \item Jedes \texttt{Device-Modul} implementiert eigene Controller, Services und Repositories, um die jeweiligen
        Anforderungen zu erfüllen. Sie können alle Funktionalitäten des Common-Moduls nutzen, aber nicht umgekehrt.
        \item Wenn ein Feature im Common-Modul gerätespezifische Daten oder Logik benötigt, wird dafür meist das
        \textit{Strategy-Pattern} verwendet. Dabei wird ein Interface im Common-Modul definiert,
        das von den Device-Modulen implementiert und in der \textit{ASP.NET Core \acrfull{di}} registriert wird.
    \end{itemize}

    Die Device Analysis ist genau so ein Feature. Es wird eine allgemeine Implementierung im Common-Modul geben.
    Die Device-Module, wenn sie dies brauchen, werden ihre eigenen Implementierungen eines im Common-Modul definierten
    Interfaces bereitstellen und sich somit im Common-Modul registrieren. Je nachdem ob eine Implementierung für ein
    Gerätetyp vorhanden ist oder nicht, wird diese oder eine Standard-Implementierung im Common-Modul verwendet.
    Dies entspricht dem \textit{Strategy-Pattern}.

    \subsection{Rest API}

    Für die Kommunikation zwischen Frontend und Backend wird ein neuer REST API Endpunkt im Common-Modul des Backends erstellt.
    Der Endpunkt wird unter dem Pfad \texttt{api/core/DeviceAnalysis/\{deviceId\}/MessageStatistics} implementiert. Dieser Pfad
    folgt dem bestehenden Muster (api/[Modul]/[Feature]/\{deviceId\}) und ist leicht verständlich. Außerdem ermöglicht dieses Schema
    eine einfache Erweiterung in der Zukunft, falls weitere Analyse-Endpunkte neben der MessageStatistics hinzugefügt werden sollen.

    Die Device-Id wird als Pfadparameter übergeben. Alle anderen Parameter werden als Query-Parameter übergeben. Das sind:
    \begin{itemize}
        \item \texttt{from}: Startdatum der Auswertung im ISO 8601 Format (z.B. 2023-01-01T00:00:00Z)
        \item \texttt{to}: Enddatum der Auswertung im ISO 8601 Format (z.B. 2023-01-07T23:59:59Z)
        \item \texttt{deviceType}: Der Typ des Geräts als \textit{string}. Dies bestimmt, welches Device-Modul die Anfrage verarbeitet. Das Backend kann den Typ aus der Device-Id nicht ableiten, ohne
        zusätzliche Abfragen in der Datenbank durchzuführen. Daher muss der Client den Typ explizit angeben.
    \end{itemize}

    Die Antwort ist ein JSON-Objekt, das die Nachrichtentypen und deren Anzahl im angegebenen Zeitraum enthält. Ein Beispiel für die Antwortstruktur ist wie folgt:
    \begin{verbatim}
    {
        "connectEvent": 150,
        "assignmentChange": 35,
        "firmwareUpdate": 20
    }
    \end{verbatim}

    \subsection{Frontend-Architektur}

    Das Frontend ist in Angular als eine \textit{Single-Page Application} (SPA) aufgebaut. Ähnlich wie im Backend gibt es Common-Module, Feature-Module und Device-Module.
    \begin{itemize}
        \item Die \texttt{Common-Module} enthalten feature- und geräteübergreifende Komponenten, Services und Utilities, die von allen anderen Modulen genutzt werden können.
        \item Die \texttt{Feature-Module} sind für nicht-gerätespezifische aber selbstständige Features zuständig, wie z.B. Suchseite, Profilseite usw.
        \item Für jeden Gerätetyp gibt es ein eigenes \texttt{Device-Modul}, das die gerätespezifischen Komponenten, Services und Routen enthält und sie bei den Feature-Modulen registriert.
        \item Für nicht-gerätespezifische Komponenten, die aber nur in den Device-Modulen verwendet werden, gibt es ein \texttt{Device-Common-Modul}. Darin werden z.B. gemeinsame UI-Komponenten
        für die Geräte-Detailansicht bereitgestellt.
    \end{itemize}

    Die Auswertungsansicht soll in jede Geräte-Detailansicht als Tab eingebaut werden. Daher wird sie einmal im Device-Common-Modul implementiert und von den Device-Modulen
    genutzt. Allerdings muss das \Gls{routing} und die Anzeigelogik von Tabs in jedem Device-Modul geändert werden, da jedes Modul seine eigenen Routen und Tabs verwaltet.
    Damit die Rollen \acrshort{pm} und \acrshort{ts} ohne Freigabe durch den Facherrichter auf die Auswertungsansicht zugreifen können, muss die Sichtbarkeit der Tabs
    in allen Device-Modulen überarbeitet werden.

    \subsection{Frontend-UI}

    Grundsätzlich sind zwei bestehende Ansichten im Frontend relevant: die Suchseite und die Geräte-Detailansicht.
    Auf den folgenden Abbildungen sind die beiden Ansichten dargestellt:

    \begin{figure}[H]
        \centering
        \includegraphics[width=0.9\textwidth]{../figures/search-screen.png}
        \caption{
            Suchseite aus Sicht eines \acrshort{ts} (mit Mock-Daten)
        }
    \end{figure}

    Beim Klicken auf eine Gerätekachel kommt man auf die Geräte-Detailansicht (vorausgesetzt, man
    hat Zugriff auf das Gerät):

    \begin{figure}[H]
        \centering
        \includegraphics[width=0.9\textwidth]{../figures/device-detail-screen.png}
        \caption{
            Geräte-Detailansicht aus Sicht eines \Gls{facherrichter}s (mit Mock-Daten)
        }
    \end{figure}

    Die Auswertungsansicht wird als ein Tab in der Geräte-Detailansicht implementiert. Der Tab enthält eine Auswahlmöglichkeit für den Zeitraum der Auswertung
    (letzte 24 Stunden, letzte 7 Tage, letzter Monat usw.) und eine Darstellung der Gerätenachrichten in einem horizontalen Balkendiagramm. An der x-Achse
    wird die Anzahl der Nachrichten und an der y-Achse der Nachrichtentyp angezeigt.

    Für einen \acrshort{pm}, der davor auf die Geräte-Detailansicht gar keinen Zugriff hatte, wird dies der einzige Tab sein, den er sehen kann. Für einen \acrshort{ts},
    der keinen Zugriff auf das Gerät hat, wird dies der einzig \textit{zugängliche} Tab sein. Andere Tabs wie Fernwartung, Logs usw. werden zwar in der Tab-Leiste angezeigt,
    leiten aber zu einer Seite weiter, die den fehlenden Zugriff erklärt. Falls der \acrshort{ts} Zugriff auf das Gerät hat, werden alle Tabs wie gewohnt angezeigt.
    Wenn ein \acrshort{ts} ohne Gerätezugriff auf einen unzugänglichen Tab klickt, wird stattdessen ein Tab angezeigt, das den fehlenden Zugriff erklärt.

    Ein Mockup der Auswertungsansicht wurde von dem UI/UX-Designer von \acrshort{abus} bereitgestellt\footnote{
        Das Mockup kann nicht zur Dokumentation beigefügt werden, da es von dem Designer gelöscht wurde, nachdem ein neues Design gefordert wurde.
        Dies wird im Kapitel \ref{sec:implementation-frontend} erläutert.}. Die Darstellung des Diagramms wurde allerdings mir überlassen, da sie
    stark von der gewählten Chart-Bibliothek abhängt.

    Die Chart-Bibliothek \textbf{Chart.js} wurde ausgewählt, da sie eine gute Balance zwischen Funktionsumfang, Einfachheit und Performance bietet. Außerdem hat sie viel Support
    von der Community und ist gut dokumentiert. Es gibt zwar einen Angular-Wrapper für Chart.js namens \textbf{ng2-charts}, aber sie bietet nichts weiteres als eine \texttt{provideCharts()}
    Methode und ein \texttt{BaseChartDirective} für die Chart-Komponenten. Dies ist ziemlich einfach auch ohne den Wrapper zu implementieren, daher wird Chart.js direkt verwendet. So
    wird eine unnötige Abhängigkeit vermieden.

\section{Implementierungsphase}

    \subsection{Backend}

    Im Backend wurde ein neuer Controller \texttt{DeviceAnalysisController} im Common-Modul erstellt, der den Endpunkt
    \texttt{api/core/DeviceAnalysis/{deviceId}/MessageStatistics} bereitstellt. Um die Feature-Implementierung einfach
    zu halten, wurde auf ein Service Layer verzichtet und die (geräte- und datenbank-unabhängige) Logik direkt im Controller implementiert.

    Der Controller nutzt die \acrshort{di}, um alle Implementierungen des im Common-Modul definierten Interfaces \texttt{IDeviceAnalysisRepository}
    zu erhalten. Jede Implementierung ist für einen bestimmten Gerätetyp zuständig und wird im jeweiligen Device-Modul registriert. Der Controller wählt die passende Implementierung
    basierend auf dem \texttt{deviceType} Query-Parameter aus, indem er die \texttt{IsRepositoryFor(deviceType)} Methode des Interfaces aufruft. Falls keine Implementierung
    für den angegebenen Gerätetyp gefunden wird, wird die Standard-Implementierung aus dem Common-Modul verwendet.

    Die Standard-Implementierung wird in der \acrshort{di} als ein sog. \textit{Keyed Singleton} registriert. Das bedeutet, dass man über das Attribut \texttt{[FromKeyedServices(key)]}
    \textit{eine bestimmte Implementierung} des Interfaces anfordern kann. So kann die Standard-Implementierung separat von den gerätespezifischen Implementierungen injiziert werden,
    ohne dass man sie in der \acrshort{di} mit ihrer eigenen Klasse registrieren muss.

    Jede Implementierung des \texttt{IDeviceAnalysisRepository} Interface ist für die Kommunikation mit der jeweiligen Datenbank zuständig.
    Derzeit gibt es neben der Standard-Implementierung nur eine gerätespezifische Implementierung für das Device-Modul \texttt{Secoris}. Die beiden
    Implementierungen nutzen dieselbe \Gls{collection} in der \Gls{mongodb}-Datenbank, fragen diese aber auf eine unterschiedliche Weise ab.

    Eine grafische Darstellung der Backend-Architektur des neuen Features ist in der Abbildung \ref{fig:backend-architecture} zu sehen.

    \subsection{Frontend} \label{sec:implementation-frontend}

    Im Frontend wurde mit dem Routing und der Anzeigelogik in den Device-Modulen begonnen, weil das der Teil war, in dem es die größte Unsicherheit gab.
    Die Implementierung der Auswertung selbst im Device-Common-Modul war relativ einfach, da es nur eine Komponente mit einem Diagramm und einer Datumsauswahl ist.
    
    Es stellte sich heraus, dass dies die richtige Entscheidung war, da es erhebliche Herausforderungen bei der Integration der Auswertungsansicht in die Geräte-Detailansicht gab.
    Die Geräte-Detailansichten waren ursprünglich nicht dafür ausgelegt, dass auch Rollen wie \acrshort{ts} und \acrshort{pm} ohne Freigabe durch den \Gls{facherrichter} auf sie zugreifen können.
    An vielen Stellen wurde Gerätezugriff als eine Vorbedingung für die Anzeige von Inhalten angenommen, was zu Problemen führte.
    Dies wurde in der Planungsphase nicht erkannt, da diese Vorbedingung nur implizit in der bestehenden Codebasis vorhanden war und nicht dokumentiert wurde.

    Es zeigte sich zudem, dass das Routing eine erhöhte Komplexität aufweist, siehe dazu Abbildung \ref{fig:frontend-routing}.
    Dabei muss man beachten, dass das Routing in Angular nicht mit einfachen if/else/switch-Anweisungen gesteuert werden kann,
    sondern mit sogenannten \textit{Guards}.
    Es gibt mehrere Arten von Guards in Angular, die jeweils zu einem bestimmten Zeitpunkt im Routing-Prozess ausgeführt werden
    und dementsprechend verschiedene Daten zur Verfügung haben. Das macht die Implementierung der komplexen Logik noch schwieriger.

    Diese Komplexität führte zu mehreren Herausforderungen während der Implementierung:
    \begin{itemize}
        \item Der Code für das Routing und die Anzeigelogik wurde sehr unübersichtlich und schwer wartbar.
        \item Es war schwierig, alle möglichen Szenarien und Randfälle zu berücksichtigen und zu testen.
        \item Neue Device-Module in der Zukunft müssen die komplexe Logik ebenfalls implementieren, was zu Fehlern führen kann
        und die Integration neuer Geräte erschwert.
        \item Die Implementierung war in der vorgegebenen Zeit nicht vollständig abschließbar.
    \end{itemize}

    Aus diesen Gründen wurde beschlossen, das initiale Konzept \textbf{nicht} weiter zu verfolgen. Stattdessen wird die Auswertungsansicht
    als ein Dialogfenster implementiert, der von einem Button in dem Drei-Punkte-Menü der Gerätekachel auf der Suchseite geöffnet wird.
    Dies kann man in der folgenden Abbildung sehen (Systemanalyse ist der deutsche Begriff für Device Analysis):

    \begin{figure}[H]
        \centering
        \includegraphics[width=0.5\textwidth]{../figures/card-screen.png}
        \caption{
            Gerätekachel mit geöffnetem Drei-Punkte-Menü und Systemanalyse-Button
        }
    \end{figure}

    Diese Änderung hat mehrere Vorteile:
    \begin{itemize}
        \item Das Menü, die Zugriffslogik und die Registrierung des Buttons sind bereits implementiert,
        sodass nur noch wenig Setup-Arbeit nötig ist.
        \item Es gibt keine Änderungen an der Geräte-Detailansicht oder dem Routing, was die Komplexität erheblich reduziert.
        \item Die Implementierung ist in der restlichen Zeit abschließbar.
    \end{itemize}

    Die einzig verbleibende Herausforderung sind die Gerätedaten. In der Geräte-Detailansicht ist eine Übersicht
    der Gerätedaten bereits in einem Sidebar vorhanden. In einem Dialogfenster dagegen nicht -
    also müssen die Gerätedaten an das Dialogfenster übergeben und dort angezeigt werden.
    In der folgenden Abbildung ist dieser Dialog dargestellt:

    \begin{figure}[H]
        \centering
        \includegraphics[width=0.9\textwidth]{../figures/dialog-screen.png}
        \caption{
            Dialogfenster mit der Auswertungsansicht
        }
    \end{figure}

    In wenigen Stunden konnte die neue Implementierung fertiggestellt werden. Die Hauptaspekte der Implementierung sind:
    \begin{itemize}
        \item Die Geräte registrieren in ihren Gerätekachel-Komponenten (siehe Quellcode \ref{code:frontend-secoris-action-registration}) eine sog. \textit{DeviceAnalysisAction}, die ein Service im Device-Common-Modul
        zur Verfügung stellt. Dies erfolgt über eine Methode, die im Quellcode \ref{code:frontend-action-registration} dargestellt ist.
        \item Der Button nimmt als Abhängigkeit ein \texttt{DeviceAnalysisProvider}, der die gerätespezifischen Daten für die Auswertungsansicht bereitstellt. Ein Beispiel dafür
        ist die Implementierung für Secoris im Quellcode \ref{code:frontend-secoris-device-analysis-provider}.
        \item Die Grafik wird wie geplant mit \texttt{Chart.js} umgesetzt.
    \end{itemize}

    \subsection{Tests}

        Die Tests sind für die Qualitätssicherung des neuen Features von entscheidender Bedeutung.
        Allerdings wird in unserem Entwicklungs-Team kein Testabdeckungsgrad vorgegeben. Es wird
        pragmatisch entschieden, welche Tests für ein Feature notwendig sind, um dessen Qualität sicherzustellen,
        ohne dabei zu viel Zeit und Ressourcen für das Schreiben und Warten von Tests aufzuwenden.
        
        Deswegen wird im Backend hauptsächlich auf Integrationstests gesetzt, um maximale Abdeckung
        bei minimalem Aufwand zu erreichen. Im Frontend werden Unit-Tests für die wichtigsten
        Komponenten und Services geschrieben, um deren Funktionalität zu überprüfen.

        \subsubsection{Backend}

        Im Backend werden Integrationstests für alle \textit{Repositories} geschrieben, um sicherzustellen, dass sie die Daten korrekt aus der Datenbank abrufen.
        Diese Tests haben die folgenden Merkmale:
        \begin{itemize}
            \item Sie sind in \textit{xUnit} geschrieben, da das Backend bereits dieses Test-Framework verwendet.
            \item Sie verwenden ein \textit{Test Container} — eine kleine, temporäre Instanz der \Gls{mongodb}-Datenbank, die nur für die Dauer der Tests läuft.
            Dadurch wird sichergestellt, dass die Tests in einer isolierten Umgebung laufen und nicht von externen Faktoren beeinflusst werden.
            \item Sie verwenden die Bibliothek \textit{Fluent Assertions}, um die Testergebnisse auf eine lesbare und verständliche Weise zu überprüfen.
            \item Um Testläufe kurz zu halten, deckt jeder Test so viele Szenarien wie möglich ab. Dies macht sie zwar nicht so isoliert wie Unit-Tests,
            aber es reduziert die Gesamtanzahl der Tests erheblich.
            \item Die Tests verwenden ein \textit{Arrange, Act, Assert (AAA)}-Muster, um die Struktur der Tests klar und konsistent zu halten.
        \end{itemize}

        Ein Beispiel für einen Integrationstest ist im Quellcode \ref{code:backend-integration-test} dargestellt.

        \subsubsection{Frontend}
        
        Als das erste Konzept für die Frontend-Implementierung verworfen wurde (siehe Abschnitt \ref{sec:implementation-frontend}),
        mussten auch viele Frontend-Tests verworfen werden, da sie auf dem ursprünglichen Konzept basierten. Nur
        eine Handvoll Tests konnte wiederverwendet werden.

        Für das neue Konzept werden aus folgenden Gründen \textbf{keine} automatisierten Tests geschrieben:
        \begin{itemize}
            \item Die Implementierung ist relativ einfach und besteht entweder aus UI-Komponenten oder aus
            Registrierungslogik, die bereits in anderen Teilen des Codes getestet wurde. Also wären hier
            \textit{Unit-Tests} nicht sehr nützlich.
            \item \textit{Integrationstests} werden im Frontend nur für geschäftskritische Features geschrieben,
            z.B. die Suchseite. Die Auswertungsansicht ist kein geschäftskritisches Feature, daher
            rechtfertigt sie keine \textit{Integrationstests}.
            \item Die für das Frontend-Testing eingeteilte Zeit wurde schon für die ursprüngliche Implementierung
            verwendet. Zeit von anderen Aufgaben abzuzweigen, um Tests zu schreiben, wäre aus oben genannten Gründen
            nicht effizient gewesen.
        \end{itemize}
\section{Abnahme und Deployment}

    \subsection{Code-Review}

    Code-Reviews fanden sowohl während der Implementierungsphase als auch nach deren Abschluss statt.
    Während der Implementierung wurden regelmäßig kleinere Code-Reviews durchgeführt, um sicherzustellen,
    dass die Entwicklung in die richtige Richtung geht und um frühzeitig Feedback zu erhalten.

    Nach Abschluss der Implementierung wurde ein \textit{Pull Request} erstellt und von zwei Entwicklern
    (einem für das Backend und einem für das Frontend) überprüft. Dabei sind keine größeren Probleme
    festgestellt worden. Es gab nur einige Anmerkungen bezüglich Code-Stil, dupliziertem Code und einigen
    kleineren Verbesserungen, die schnell behoben wurden.

    \subsection{Deployment}

    Das \acrshort{portal} hat drei Umgebungen: Dev, Stage und Prod (Produktion). Auf Dev sind meist nur
    Entwickler aktiv, die dort neue Features implementieren und testen. Auf Stage können auch bestimmte
    Mitarbeiter von \acrshort{abus} testen, ob alles wie erwartet funktioniert. Prod ist die Live-Umgebung,
    die von den Endbenutzern genutzt wird.

    Nachdem die Implementierung abgeschlossen war, lokal getestet wurde und das Code-Review bestanden hat,
    wurde die neue Version des Backends und Frontends gebaut, getestet und auf die Dev-Umgebung
    deployed. Dort wurde nochmal manuell getestet, ob alles wie erwartet funktioniert.

    An dieser Stelle wurde ein Bug gefunden. Dies wird in der nächsten Sektion beschrieben. Nachdem der Bug
    behoben war, wurde die neue Version nochmal auf die Dev-Umgebung deployed und getestet. Diesmal
    funktionierte alles wie erwartet und der Build wurde auf die Staging-Umgebung deployed.

    Auf die Prod-Umgebung wird das neue Feature erst mit dem nächsten Release deployed, da jedes Deployment
    auf Prod ein Release darstellt und dies darf nur zu bestimmten, von \acrshort{abus} eingeplanten Terminen
    durchgeführt werden.

    \subsection{Bugfixes}

    Als das neue Feature in der Dev-Umgebung getestet wurde, wurde ein Bug gefunden: \\
    Das Diagramm konnte nicht geladen werden, da die API-Anfrage an das Backend mit einem 400-Fehler
    (Bad Request) fehlschlug. Nach einer Untersuchung des Problems wurde festgestellt, dass
    das Backend die Anfrage ablehnte, weil das Enddatum in der Zukunft lag.
    Dies lag daran, dass das Backend und das Frontend unterschiedliche Zeitzonen verwendeten.
    Das Backend verwendete UTC (Coordinated Universal Time), während das Frontend die lokale
    Zeitzone des Benutzers verwendete (in diesem Fall UTC+1).

    Das Backend wurde initial so implementiert, dass es eine Fehlerantwort zurückgibt,
    wenn das Enddatum in der Zukunft liegt. Dies war eine bewusste Entscheidung,
    um ungültige Anfragen abzuweisen. Allerdings wurde nach einer genaueren Überlegung gesehen,
    dass dieses Verhalten kein echtes Problem löst und stattdessen den Endpunkt bloß weniger
    robust macht. Daher wurde beschlossen, dass das Enddatum im Backend in solchen Fällen
    automatisch auf das aktuelle Datum gesetzt wird.

    Dies hat den Bug behoben und nach einem erneuten Deployment auf die Dev-Umgebung
    funktionierte das Diagramm wie erwartet. Weitere Bugs wurden nicht gefunden.

    \subsection{Abnahme}

    Nach einer Demonstration des neuen Features für den Projektbetreuer von \acrshort{abus}
    wurde eine kleine Änderung am Design des Diagramms gewünscht. Die Standartoption für die Zeitauswahl
    sollte auf \enquote{Letzte 7 Tage} anstatt \enquote{Letztes Jahr} gesetzt werden und eine Option für \enquote{Letzte 24 Stunden}
    sollte hinzugefügt werden. Diese Änderung wurde kurzfristig implementiert und wieder auf Dev und Stage deployed.

    Danach wurde das neue Feature von dem Projektbetreuer abgenommen. Es gab keine weiteren Änderungswünsche
    und das Feature wurde für gut befunden.

    \subsection{Dokumentation}

    Als Entwicklerdokumentation im Backend dient die automatisch generierte \texttt{Swagger}-Dokumentation
    für die Backend-API. Diese ist unter \url{https://stage.abus-cloud.com/swagger/index.html} erreichbar
    (siehe Core - \texttt{POST /api/core/DeviceAnalysis/\{deviceId\}/MessageStatistics}). Außerdem wurde auf
    sprechende Methoden- und Variablennamen geachtet, damit der Code auch ohne zusätzliche Dokumentation
    verständlich ist.
    
    Im Frontend wurde ebenfalls auf sprechende Namen geachtet. Da das Frontend jedoch relativ einfach und
    selbsterklärend ist, wurde keine zusätzliche Dokumentation erstellt.

    Kundendokumentation ist in Form des abgeschlossenen \textit{Product Backlog Item} in Jira vorhanden. Außerdem wurden zwei
    Ansprechpartner von \acrshort{abus} eingewiesen, die bei Fragen kontaktiert werden können. Das Feature
    wird ausschließlich von ABUS-internen Rollen genutzt, daher ist keine weitere Kundendokumentation erforderlich.
\section{Fazit}

    \subsection{Soll-/Ist-Vergleich}

    \lipsum[1]

    \subsection{Erkenntnisse}

    \lipsum[1]

    \subsection{Ausblick}

    \lipsum[1]

\pagebreak
\appendix
\renewcommand{\thesection}{\Alph{section}}

\section{Literatur}
\renewcommand{\refname}{}
\vspace{-1.5em}
\bibliographystyle{alpha}
\bibliography{../bib/sample}

\pagebreak
\section{Nutzung der künstlichen Intelligenz}

    Bei der Implementierung und Dokumentation des Projekts wurden verschiedene
    KI-Tools eingesetzt, um den Entwicklungsprozess zu unterstützen und zu beschleunigen.
    Die verwendeten Tools waren:
    \begin{itemize}
        \item ChatGPT (OpenAI, \url{https://chatgpt.com/}, letzter Aufruf: 12.12.2025)
        \item GitHub Copilot (GitHub, \url{https://github.com/copilot}, letzter Aufruf: 12.12.2025, hauptsächlich als VSCode-Plugin)
    \end{itemize}

    \subsection{Verwendungszweck bzw. Einsatzszenario}
    
    Die KI-Tools wurden hauptsächlich für folgende Zwecke eingesetzt:
    \begin{itemize}
        \item IntelliSense-ähnliche Code-Vervollständigungen (kein \enquote{Vibe Coding}, reine Tipphilfe)
        \item Formulierungshilfen für die Dokumentation (aber \textbf{nicht} die Inhalte selbst)
        \item Formatierungshilfen (z.B. LaTeX-Syntax)
        \item Generierung von Mermaid-Diagrammen für die Dokumentation
        (aus detaillierten textuellen Beschreibungen)
    \end{itemize}

    \subsection{Ergänzende Hinweise}

    Es ist kaum möglich, den genauen Anteil der mit KI-Tools erstellten Inhalte zu beziffern.
    Genauso wenig ist es möglich, genaue Stellenangaben in der Arbeit zu machen, da die
    KI-Tools nicht zur Erstellung ganzer Abschnitte oder Kapitel verwendet wurden.
    Sie dienten als Hilfsmittel während des ganzen Projekts.
    
    \textbf{Es ist jedoch sicher, dass die Eigenständigkeit der Arbeit gewahrt bleibt, da alle
    Inhalte und Entscheidungen ausschließlich von mir stammen. Die KI-Tools wurden nur als eine
    Arbeitserleichterung bzw. \textit{Quality of Life}-Verbesserung eingesetzt.}

\pagebreak
\renewcommand{\lstlistingname}{Quellcode}

\section{Abbildungen}
\begin{figure}[H]
    \centering
    \includegraphics[width=0.9\textwidth]{../figures/system-architecture.png}
    \caption{
        Systemübersicht des \acrshort{portal}s
    }
    \label{fig:system-architecture}
\end{figure}
\begin{figure}[H]
    \centering
    \includegraphics[width=0.9\textwidth]{../figures/backend-architecture.png}
    \caption{
        Backend-Architektur des neuen Features (vereinfacht dargestellt)
    }
    \label{fig:backend-architecture}
\end{figure}


\begin{figure}[H]
    \centering
    \includegraphics[width=0.9\textwidth]{../figures/frontend-routing.png}
    \caption{
        Routing-Entscheidungsbaum für den Auswertungs-Tab in einem Device-Modul
    }
    \label{fig:frontend-routing}
\end{figure}

\clearpage
\section{Code-Beispiele}
\definecolor{lightgray}{rgb}{.9,.9,.9}
\definecolor{darkgray}{rgb}{.4,.4,.4}
\definecolor{purple}{rgb}{0.65, 0.12, 0.82}

\lstdefinelanguage{JavaScript}{
  keywords={typeof, new, true, false, catch, function, return, null, catch, switch, var, if, in, while, do, else, case, break},
  keywordstyle=\color{blue}\bfseries,
  ndkeywords={class, export, boolean, throw, implements, import, this},
  ndkeywordstyle=\color{darkgray}\bfseries,
  identifierstyle=\color{black},
  sensitive=false,
  comment=[l]{//},
  morecomment=[s]{/*}{*/},
  commentstyle=\color{purple}\ttfamily,
  stringstyle=\color{red}\ttfamily,
  morestring=[b]',
  morestring=[b]"
}

\lstset{
   language=JavaScript,
   backgroundcolor=\color{lightgray},
   extendedchars=true,
   basicstyle=\footnotesize\ttfamily,
   showstringspaces=false,
   showspaces=false,
   numbers=left,
   numberstyle=\footnotesize,
   numbersep=9pt,
   tabsize=2,
   breaklines=true,
   showtabs=false,
   captionpos=b
}

\textbf{Anmerkung:} Die folgenden Codebeispiele sind Auszüge aus dem tatsächlichen Quellcode.

Allerdings wurden deutschsprachige Kommentare und Erklärungen hinzugefügt, um den Code ohne
weiteren Kontext verständlich zu machen. Im tatsächlichen Quellcode sind diese Kommentare
nicht vorhanden.

Außerdem wurden einige für das Projekt nicht relevante Teile des Codes mit
ihren Imports und Abhängigkeiten weggelassen, um den Leser nicht mit irrelevanten Details
zu überladen. Einige Klassen- und Variablennamen wurden verkürzt/vereinfacht.

Umlaute in den Kommentaren wurden aus technischen Gründen entfernt (z.B. \enquote{ä} zu \enquote{ae}).

\lstinputlisting[language=JavaScript, float=h, caption={Registrierung des \texttt{DeviceAnalysisDeviceActions} in der Secoris-Gerätekachel}, label={code:frontend-secoris-action-registration}]{../code/secoris-search-result-card.component.ts}

\vspace{1em}

\lstinputlisting[language=JavaScript, float=h, caption={Provider von \texttt{DeviceActions}}, label={code:frontend-action-registration}]{../code/device-actions.provider.ts}

\vspace{1em}

\lstinputlisting[language=JavaScript, float=h, caption={Implementierung des \texttt{DeviceAnalysisProvider} für Secoris}, label={code:frontend-secoris-device-analysis-provider}]{../code/secoris-device-analysis.provider.ts}

\vspace{1em}

\lstinputlisting[language={[Sharp]C}, float=h, caption={Integrationstest der Default-Implementierung des \texttt{IDeviceAnalysisRepository}}, label={code:backend-integration-test}]{../code/DefaultDeviceAnalysisRepositoryTests.cs}


\end{document}