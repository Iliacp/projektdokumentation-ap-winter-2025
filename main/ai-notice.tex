\section{Nutzung der künstlichen Intelligenz}

    Bei der Implementierung und Dokumentation des Projekts wurden verschiedene
    KI-Tools eingesetzt, um den Entwicklungsprozess zu unterstützen und zu beschleunigen.
    Die verwendeten Tools waren:
    \begin{itemize}
        \item ChatGPT (OpenAI, \url{https://chatgpt.com/}, letzter Aufruf: 12.12.2025)
        \item GitHub Copilot (GitHub, \url{https://github.com/copilot}, letzter Aufruf: 12.12.2025, hauptsächlich als VSCode-Plugin)
    \end{itemize}

    \subsection{Verwendungszweck bzw. Einsatzszenario}
    
    Die KI-Tools wurden hauptsächlich für folgende Zwecke eingesetzt:
    \begin{itemize}
        \item IntelliSense-ähnliche Code-Vervollständigungen (kein \enquote{Vibe Coding}, reine Tipphilfe)
        \item Formulierungshilfen für die Dokumentation (aber \textbf{nicht} die Inhalte selbst)
        \item Formatierungshilfen (z.B. LaTeX-Syntax)
        \item Generierung von Mermaid-Diagrammen für die Dokumentation
        (aus detaillierten textuellen Beschreibungen)
    \end{itemize}

    \subsection{Ergänzende Hinweise}

    Es ist kaum möglich, den genauen Anteil der mit KI-Tools erstellten Inhalte zu beziffern.
    Genauso wenig ist es möglich, genaue Stellenangaben in der Arbeit zu machen, da die
    KI-Tools nicht zur Erstellung ganzer Abschnitte oder Kapitel verwendet wurden.
    Sie dienten als Hilfsmittel während des ganzen Projekts.
    
    \textbf{Es ist jedoch sicher, dass die Eigenständigkeit der Arbeit gewahrt bleibt, da alle
    Inhalte und Entscheidungen ausschließlich von mir stammen. Die KI-Tools wurden nur als eine
    Arbeitserleichterung bzw. \textit{Quality of Life}-Verbesserung eingesetzt.}