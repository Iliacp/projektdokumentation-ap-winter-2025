\section{Analysephase}

    \subsection{Wirtschaftlichkeitsanalyse} \label{sec:profitability-analysis}

        Für die conplement AG lohnt sich das Projekt schon deswegen, weil im Vertrag
        mit der \acrshort{abus} zusätzlich zu den Projektkosten auch ein Gewinnaufschlag und
        ein Risikozuschlag vereinbart wurde (die genaue Höhe ist vertraulich). Somit sind die
        Projektkosten gedeckt und es wird zusätzlich ein Gewinn erzielt.

        \subsubsection{Projektkosten}

        Die Projektkosten setzen sich aus den Personalkosten der beteiligten Mitarbeiter
        der conplement AG zusammen. Alle verwendeten Ressourcen sind bereits im Stundensatz
        der Mitarbeiter einkalkuliert, sodass keine zusätzlichen Kosten anfallen.

        Für einen Azubi (mich) ist ein Stundensatz von 35€ veranschlagt, während für
        einen erfahrenen Softwareentwickler ein Stundensatz von 104€ gilt. Das Projekt
        umfasst insgesamt 80 Stunden, die sich wie folgt aufteilen:

        \begin{itemize}
            \item 80 Stunden Arbeit von mir als Azubi
            \item 5 Stunden Arbeit von einem erfahrenen Softwareentwickler zur
            Unterstützung bei der Planung und Code-Reviews
        \end{itemize}

        Daraus ergibt sich die folgende Berechnung der Projektkosten:

        \begin{table}[H]
            \centering
            \begin{tabular}{l r r r}
                \toprule
                \textbf{Rolle} & \textbf{Stundensatz} & \textbf{Stunden} & \textbf{Gesamtkosten} \\
                \midrule
                Azubi & 35€ & 80 & 2.800€ \\
                Erfahrener Entwickler & 104€ & 5 & 520€ \\
                \midrule
                \textbf{Gesamt} & & & \textbf{3.320€} \\
                \bottomrule
            \end{tabular}
            \caption{Projektkostenübersicht}
        \end{table}

        \subsubsection{Amortisationsrechnung aus Kundensicht}

        Genaue Zahlen zur Amortisation des Projekts sind schwer zu ermitteln,
        da das Projekt keinen direkten Umsatz für \acrshort{abus} generiert.
        Stattdessen spart das Projekt indirekt Kosten, indem es die Effizienz des
        \acrshort{ts} verbessert.

        Basierend auf Gesprächen mit der \acrshort{abus} und Annahmen über die
        Arbeitsweise des \acrshort{ts} lassen sich folgende Schätzungen treffen:

        \begin{itemize}
            % \item Ein \acrshort{ts} arbeitet 35 Stunden pro Woche. - nicht relevant
            \item Der Stundensatz eines \acrshort{ts} beträgt ca. 90€ (angenommen).
            \item Durch das neue Feature wird 30 Minuten weniger für eine Support-Anfrage
            benötigt (angenommen).
            \item Bei ungefähr 10\% aller Geräte wird pro Jahr eine Support-Anfrage gestellt,
            die durch das neue Feature vereinfacht wird (angenommen).
            \item Es sind insgesamt 6.572 Geräte im \acrshort{portal} registriert (Stand: Dezember 2025).
        \end{itemize}

        Daraus ergeben sich folgende Berechnungen:

        \begin{itemize}
            \item Anzahl der Support-Anfragen pro Jahr:
            \[
                6.572 \text{ Geräte} \times 10\% = 657{,}2 \approx 657 \text{ Anfragen}
            \]
            \item Zeitersparnis pro Jahr durch das neue Feature:
            \[
                657 \text{ Anfragen} \times 0{,}5 \text{ Stunden} = 328{,}5 \text{ Stunden}
            \]
            \item Kosteneinsparung pro Jahr:
            \[
                328,5 \text{ Stunden} \times 90\text{€}/\text{Stunde} = 29.565\text{€}
            \]
            \item Kosteneinsparung pro Monat:
            \[
                29.565\text{€} / 12 = 2.463{,}75\text{€}
            \]
        \end{itemize}

        Also wird sich das Projekt für die \acrshort{abus} bereits nach etwa
        \[
            3.320\text{€} / 2.463{,}75\text{€} \approx 1{,}35 \text{ Monate} \approx 2 \text{ Monate}
        \]
        amortisieren, wenn die Annahmen zutreffen.
        
        Aber auch bei Abweichungen von diesen Annahmen ist es sehr wahrscheinlich,
        dass sich das Projekt innerhalb weniger Monate amortisiert. Z.B. wenn nur für 5\%
        der Geräte pro Jahr eine Support-Anfrage gestellt wird, bei der das neue Feature nur 15
        Minuten spart und der Stundensatz eines \acrshort{ts} nur 75€ beträgt, würde sich
        das Projekt immer noch innerhalb von etwa einem halben Jahr amortisieren.
        
        \subsubsection{Nicht-monetäre Vorteile für den Kunden}

        Es gibt auch mehrere nicht-monetäre bzw. schwer berechenbare Vorteile für die \acrshort{abus}
        durch das neue Feature:

        \begin{itemize}
            \item Weniger Freigabeanfragen und somit mehr Zufriedenheit bei \Gls{endnutzer} und \Gls{facherrichter}
            \item Verringerte Arbeitslast der \acrshort{ts} und \acrshort{pm}
            \item Weniger Support-Anfragen, die an das Entwicklungsteam eskalieren
            \item Neue Sichtweise auf die Gerätedaten und Erkenntnisse, die vorher nicht möglich waren
        \end{itemize}

    \subsection{Ist-Analyse}

    Das vorhandene System besteht aus einem Backend, einem Frontend und einer \Gls{mongodb}-Datenbank.
    Das Backend ist mit \textit{ASP.NET} implementiert und stellt eine \textit{REST-API} bereit, die vom
    \textit{Angular}-Frontend genutzt wird. Das Backend hat außerdem viele Schnittstellen zu weiteren
    internen und externen Systemen, wie z.B. einen \textit{Service Bus}, ein \textit{IoT Hub}, Systeme von
    \acrshort{abus} usw. Das Frontend wird von einem \textit{Node.js}-Server an die Clients ausgeliefert, der
    gleichzeitig auch als ein \textit{Proxy} für die API-Anfragen fungiert.
    Eine abstrakte Übersicht der Architektur ist in der Abbildung \ref{fig:system-architecture} dargestellt.

    Die \Gls{mongodb}-Datenbank speichert alle Gerätedaten, darunter auch die
    Gerätenachrichten, die für die Auswertung benötigt werden. Sie sind in einer sog.
    \Gls{collection} gespeichert, die den Namen \texttt{common-messages} trägt. Jedes Dokument in
    dieser \Gls{collection} repräsentiert eine einzelne Gerätenachricht und hat folgendes Format
    (nur relevante Felder dargestellt):

    \begin{verbatim}
    {
        deviceId: Guid,             // ID des Geräts, das die Nachricht gesendet hat
        messageType: "string",      // Typ der Nachricht (z.B. "connectEvent")
        deviceType: "string",       // Typ des Geräts (z.B. "Secoris")
        enqueueTimeUtc: Date,       // Eingangszeit der Nachricht in UTC
        rawPayload: object          // Rohdaten der Nachricht
        ...                         // weitere Felder
    }
    \end{verbatim}

    Grundsätzlich soll als Nachrichtenart der \texttt{messageType} ausgewertet werden.
    Allerdings ist dieser Typ nicht immer aussagekräftig genug, da manche Gerätetypen
    auch viel detailliertere Typen in den \texttt{rawPayload}-Daten haben. Daher
    muss die genaue Auswertung je nach Gerätetyp unterschiedlich implementiert werden.