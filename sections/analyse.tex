\section{Analysephase}

    \subsection{Ist-Analyse}

    todo

    \subsection{Wirtschaftlichkeitsanalyse}

        Für die Conplement AG lohnt sich das Projekt schon deswegen, weil im Vertrag
        mit der \acrshort{abus} zusätzlich zu den Projektkosten auch ein Gewinnaufschlag
        vereinbart wurde (die genaue Höhe ist vertraulich). Somit sind die Projektkosten
        gedeckt und es wird zusätzlich ein Gewinn erzielt.

        \subsubsection{Projektkosten}

        Die Projektkosten setzen sich aus den Personalkosten der beteiligten Mitarbeiter
        der Conplement AG zusammen. Alle verwendeten Ressourcen sind bereits im Stundensatz
        der Mitarbeiter einkalkuliert, sodass keine zusätzlichen Kosten anfallen.

        Für einen Azubi (mich) ist ein Stundensatz von 35€ veranschlagt, während für
        einen erfahrenen Softwareentwickler ein Stundensatz von 104€ gilt. Das Projekt
        umfasst insgesamt 80 Stunden, die sich wie folgt aufteilen:

        \begin{itemize}
            \item 75 Stunden Arbeit von mir als Azubi
            \item 5 Stunden Arbeit von einem erfahrenen Softwareentwickler zur
            Unterstützung bei der Planung und Code-Reviews
        \end{itemize}

        Daraus ergibt sich die folgende Berechnung der Projektkosten:

        \begin{table}[H]
            \centering
            \begin{tabular}{l r r r}
                \toprule
                \textbf{Rolle} & \textbf{Stundensatz} & \textbf{Stunden} & \textbf{Gesamtkosten} \\
                \midrule
                Azubi & 35€ & 75 & 2.625€ \\
                Erfahrener Entwickler & 104€ & 5 & 520€ \\
                \midrule
                \textbf{Gesamt} & & & \textbf{3.145€} \\
                \bottomrule
            \end{tabular}
            \caption{Projektkostenübersicht}
        \end{table}

        \subsubsection{Amortisationsrechnung aus Kundensicht}

        Genaue Zahlen zur Amortisation des Projekts sind schwer zu ermitteln,
        da das Projekt keinen direkten Umsatz für \acrshort{abus} generiert.
        Stattdessen spart das Projekt indirekt Kosten, indem es die Effizienz der
        \acrshort{ts} verbessert.

        Basierend auf Gesprächen mit der \acrshort{abus} und Annahmen über die
        Arbeitsweise der \acrshort{ts} lassen sich folgende Schätzungen treffen:

        \begin{itemize}
            % \item Ein \acrshort{ts} arbeitet 35 Stunden pro Woche. - nicht relevant
            \item Der Stundensatz eines \acrshort{ts} beträgt ca. 90€ (angenommen).
            \item Durch das neue Feature wird 30 Minuten weniger für eine Support-Anfrage
            benötigt (angenommen).
            \item Bei ungefähr 10\% aller Geräte wird pro Jahr eine Support-Anfrage gestellt,
            die durch das neue Feature vereinfacht wird (angenommen).
            \item Es sind insgesamt 6572 Geräte im \acrshort{portal} registriert (Stand: December 2025).
        \end{itemize}

        Daraus ergeben sich folgende Berechnungen:

        \begin{itemize}
            \item Anzahl der Support-Anfragen pro Jahr:
            \[
                6572 \text{ Geräte} \times 10\% = 657,2 \approx 657 \text{ Anfragen}
            \]
            \item Zeitersparnis pro Jahr durch das neue Feature:
            \[
                657 \text{ Anfragen} \times 0,5 \text{ Stunden} = 328,5 \text{ Stunden}
            \]
            \item Kosteneinsparung pro Jahr:
            \[
                328,5 \text{ Stunden} \times 90\text{€}/\text{Stunde} = 29.565\text{€}
            \]
            \item Kosteneinsparung pro Monat:
            \[
                29.565\text{€} / 12 = 2.463,75\text{€}
            \]
        \end{itemize}

        Also wird sich das Projekt für die \acrshort{abus} bereits nach etwa
        \[
            3.145\text{€} / 2.463,75\text{€} \approx 1,28 \text{ Monate} \approx 2 \text{ Monate}
        \]
        amortisieren, wenn die Annahmen zutreffen.
        
        Aber auch bei Abweichungen von diesen Annahmen ist es sehr wahrscheinlich,
        dass sich das Projekt innerhalb weniger Monate amortisiert. Z.B. wenn nur 5\%
        der Geräte pro Jahr eine Support-Anfrage stellen, bei der das neue Feature nur 15
        Minuten spart, und der Stundensatz eines \acrshort{ts} nur 75€ beträgt, würde sich
        das Projekt immer noch innerhalb von etwa einem halben Jahr amortisieren.
        
        \subsubsection{Nicht-monetäre Vorteile für den Kunden}

        todo 

        % Ein sehr großer Vorteil des zu implementierenden Features ist die verminderte
        % Arbeitslast der \acrshort{pm} und der Entwickler unseres Entwicklungs-Teams.
        % Da die \acrshort{ts} nun selbstständig einfache Analysen der Gerätedaten durchführen können,
        % müssen sie nicht mehr so oft die \acrshort{pm} oder die Entwickler um Hilfe bitten.
        % Dies spart nicht nur Geld, sondern öffnet auch Kapazitäten für die \acrshort{pm} und die Entwickler,
        % sodass sie sich auf wichtigere Aufgaben konzentrieren können.

        % Außerdem müssen die \acrshort{ts} mit dem neuen Feature nicht mehr so oft eine
        % Zugriffsanfrage an den \Gls{facherrichter} stellen, um technische Probleme zu identifizieren.
        % Dies verbessert die Effizienz der \acrshort{ts} und die Zufriedenheit der \Gls{endnutzer}
        % und \Gls{facherrichter}, da ihr Problem schneller gelöst werden kann.