\definecolor{lightgray}{gray}{0.9}
\definecolor{lightgreen}{cmyk}{0.43, 0, 0.79, 0}

\section{Fazit}

    \subsection{Soll-/Ist-Vergleich}

    \begin{table}[H]
    \centering
    \begin{tabular}{|l|c|c|c|}
    \hline
    \textbf{Phase} & \textbf{Geplante Zeit (Stunden)} & \textbf{Tatsächliche Zeit (Stunden)} & \textbf{Differenz (Stunden)} \\
    \hline
    \rowcolor{lightgray} \textbf{1} & \textbf{12} & \textbf{11} & \textbf{-1} \\
    1.1 & 2 & 2 &  \\
    1.2 & 3 & 2 & -1 \\
    1.3 & 1 & 2 & +1 \\
    1.4 & 5 & 4 & -1 \\
    1.5 & 1 & 1 &  \\
    \rowcolor{lightgray} \textbf{2} & \textbf{50} & \textbf{52} & \textbf{+2} \\
    2.1 & 18 & 26:30 & +8:30 \\
    2.2 & 5 & 5 &  \\
    2.3 & 18 & 14:30 & -3:30 \\
    2.4 & 8 & 5 & -3 \\
    2.5 & 1 & 1 &  \\
    \rowcolor{lightgray} \textbf{3} & \textbf{5} & \textbf{4} & \textbf{-1} \\
    3.1 & 2 & 2 &  \\
    3.2 & 2 & 1 & -1 \\
    3.3 & 1 & 1 &  \\
    \rowcolor{lightgray} \textbf{4} & \textbf{13} & \textbf{13} &  \\
    4.1 & 10 & 10 &  \\
    4.2 & 3 & 3 &  \\
    \rowcolor{lightgreen} \textbf{Insg.} & \textbf{80} & \textbf{80} &  \\
    \hline
    \end{tabular}
    \caption{Soll-/Ist-Vergleich der Projektphasen}
    \end{table}

    Das Zeitbudget vom 80 Stunden wurde wie geplant eingehalten. Es gab jedoch Abweichungen in den einzelnen Phasen, am meisten in
    der Implementierungsphase. Die Frontend-Implementierung hat deutlich mehr Zeit in Anspruch genommen als geplant,
    während die Backend-Implementierung weniger Zeit benötigte.

    Dafür gab es zwei Hauptgründe. Zum einen gab es Herausforderungen bei der Implementierung der Auswertungsansicht im Frontend.
    Erst tief in der Implementierung wurde klar, dass die initiale Idee nicht sinnvoll umsetzbar war und eine alternative Lösung gewählt werden musste.
    Dadurch entstand Mehrarbeit, die nicht eingeplant war.

    Der zweite Grund für die Abweichung war die Annahme, dass ich viel mehr Erfahrung im Frontend habe als im Backend und daher
    für das Backend mehr Zeit brauchen würde. Es zeigte sich jedoch, dass die Backend-Implementierung relativ unkompliziert war.
    Dadurch benötigte ich dort weniger Zeit, als für die Auswertungsansicht im \acrshort{portal} ursprünglich vorgesehen war —
    trotz meiner geringeren Erfahrung im Vergleich zum Frontend.

    \subsection{Erkenntnisse}

    In diesem Projekt war die Entscheidung, das ursprüngliche Konzept für die Auswertungsansicht zu verwerfen und
    stattdessen einen Dialog zu verwenden, entscheidend für den erfolgreichen Abschluss der Implementierung innerhalb des Zeitrahmens.
    Hätte ich weiter an dem ursprünglichen Konzept festgehalten, wäre die Implementierung wahrscheinlich nicht rechtzeitig fertig geworden.
    In der Zukunft sollte ich am besten noch früher im Prozess solche unnötig komplexen Konzepte erkennen und
    auf deren Vereinfachung bestehen.

    Außerdem hat sich die Entscheidung, \texttt{ng2-charts} nicht zu verwenden, als richtig erwiesen. Nach dem Abschluss der Implementierung unterstützte die Wrapper-Bibliothek
    die letzten zwei Versionen von Angular nicht mehr vollständig, was ein Anzeichen für mangelnde Wartung ist. Hätte ich die Bibliothek verwendet, würden in der Zukunft
    wahrscheinlich Kompatibilitätsprobleme beim Upgrade auf neuere Angular-Versionen auftreten. \texttt{Chart.js} dagegen hängt gar nicht von Angular ab und ist daher zukunftssicherer.

    \subsection{Ausblick}

    Das neue Feature zur Auswertung der Gerätedaten wird mit dem nächsten Release im ersten Quartal 2026 ausgeliefert.
    Die Ansprechpartner von \acrshort{abus}, die gleichzeitig die Nutzer davon sind, haben das Feature bereits getestet
    und für gut befunden, was auf eine erfolgreiche Einführung schließen lässt. Das Feature wurde so gebaut, dass es
    in der Zukunft leicht erweitert werden kann, z.B. durch Hinzufügen neuer Diagrammtypen oder Filteroptionen, auch
    gerätespezifisch.