\section{Abnahme und Deployment}

    \subsection{Code-Review}

    Code-Reviews fanden sowohl während der Implementierungsphase als auch nach deren Abschluss statt.
    Während der Implementierung wurden regelmäßig kleinere Code-Reviews durchgeführt, um sicherzustellen,
    dass die Entwicklung in die richtige Richtung geht und um frühzeitig Feedback zu erhalten.

    Nach Abschluss der Implementierung wurde ein \textit{Pull Request} erstellt und von zwei Entwicklern
    (einem für das Backend und einem für das Frontend) überprüft. Dabei sind keine größeren Probleme
    festgestellt worden. Es gab nur einige Anmerkungen bezüglich Code-Stil, dupliziertem Code und einigen
    kleineren Verbesserungen, die schnell behoben wurden.

    \subsection{Deployment}

    Das \acrshort{portal} hat drei Umgebungen: Dev, Stage und Prod (Produktion). Auf Dev sind meist nur
    Entwickler aktiv, die dort neue Features implementieren und testen. Auf Stage können auch bestimmte
    Mitarbeiter von \acrshort{abus} testen, ob alles wie erwartet funktioniert. Prod ist die Live-Umgebung,
    die von den Endbenutzern genutzt wird.

    Nachdem die Implementierung abgeschlossen war, lokal getestet wurde und das Code-Review bestanden hat,
    wurde die neue Version des Backends und Frontends gebaut, getestet und auf die Dev-Umgebung
    deployed. Dort wurde nochmal manuell getestet, ob alles wie erwartet funktioniert.

    An dieser Stelle wurde ein Bug gefunden. Dies wird in der nächsten Sektion beschrieben. Nachdem der Bug
    behoben war, wurde die neue Version nochmal auf die Dev-Umgebung deployed und getestet. Diesmal
    funktionierte alles wie erwartet und der Build wurde auf die Staging-Umgebung deployed.

    Auf die Prod-Umgebung wird das neue Feature erst mit dem nächsten Release deployed, da jedes Deployment
    auf Prod ein Release darstellt und dies darf nur zu bestimmten, von \acrshort{abus} eingeplanten Terminen
    durchgeführt werden.

    \subsection{Bugfixes}

    Als das neue Feature in der Dev-Umgebung getestet wurde, wurde ein Bug gefunden: \\
    Das Diagramm konnte nicht geladen werden, da die API-Anfrage an das Backend mit einem 400-Fehler
    (Bad Request) fehlschlug. Nach einer Untersuchung des Problems wurde festgestellt, dass
    das Backend die Anfrage ablehnte, weil das Enddatum in der Zukunft lag.
    Dies lag daran, dass das Backend und das Frontend unterschiedliche Zeitzonen verwendeten.
    Das Backend verwendete UTC (Coordinated Universal Time), während das Frontend die lokale
    Zeitzone des Benutzers verwendete (in diesem Fall UTC+1).

    Das Backend wurde initial so implementiert, dass es eine Fehlerantwort zurückgibt,
    wenn das Enddatum in der Zukunft liegt. Dies war eine bewusste Entscheidung,
    um ungültige Anfragen abzuweisen. Allerdings wurde nach einer genaueren Überlegung gesehen,
    dass dieses Verhalten kein echtes Problem löst und stattdessen den Endpunkt bloß weniger
    robust macht. Daher wurde beschlossen, dass das Enddatum im Backend in solchen Fällen
    automatisch auf das aktuelle Datum gesetzt wird.

    Dies hat den Bug behoben und nach einem erneuten Deployment auf die Dev-Umgebung
    funktionierte das Diagramm wie erwartet. Weitere Bugs wurden nicht gefunden.

    \subsection{Abnahme}

    Nach einer Demonstration des neuen Features für den Projektbetreuer von \acrshort{abus}
    wurde eine kleine Änderung am Design des Diagramms gewünscht. Die Standartoption für die Zeitauswahl
    sollte auf \enquote{Letzte 7 Tage} anstatt \enquote{Letztes Jahr} gesetzt werden und eine Option für \enquote{Letzte 24 Stunden}
    sollte hinzugefügt werden. Diese Änderung wurde kurzfristig implementiert und wieder auf Dev und Stage deployed.

    Danach wurde das neue Feature von dem Projektbetreuer abgenommen. Es gab keine weiteren Änderungswünsche
    und das Feature wurde für gut befunden.

    \subsection{Dokumentation}

    Als Entwicklerdokumentation im Backend dient die automatisch generierte \texttt{Swagger}-Dokumentation
    für die Backend-API. Diese ist unter \url{https://stage.abus-cloud.com/swagger/index.html} erreichbar
    (siehe Core - \texttt{POST /api/core/DeviceAnalysis/\{deviceId\}/MessageStatistics}). Außerdem wurde auf
    sprechende Methoden- und Variablennamen geachtet, damit der Code auch ohne zusätzliche Dokumentation
    verständlich ist.
    
    Im Frontend wurde ebenfalls auf sprechende Namen geachtet. Da das Frontend jedoch relativ einfach und
    selbsterklärend ist, wurde keine zusätzliche Dokumentation erstellt.

    Kundendokumentation ist in Form des abgeschlossenen \textit{Product Backlog Item} in Jira vorhanden. Außerdem wurden zwei
    Ansprechpartner von \acrshort{abus} eingewiesen, die bei Fragen kontaktiert werden können. Das Feature
    wird ausschließlich von ABUS-internen Rollen genutzt, daher ist keine weitere Kundendokumentation erforderlich.