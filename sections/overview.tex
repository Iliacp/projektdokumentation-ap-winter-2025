\section{Einleitung}

    \subsection{Projektumfeld}

    Die conplement AG ist ein mittelständischer IT-Dienstleister mit ca. 130 Mitarbeitern, zwei Standorten
    in Deutschland (Nürnberg und Dortmund) und fast 20 Jahren Markterfahrung. Sie bietet individuelle
    Lösungen sowie Beratung in Bereichen wie Web, Embedded und Mobile.
    
    Dazu gehört das \acrfull{portal} für den Kunden \acrfull{abus}. \acrshort*{abus} ist eine Tochtergesellschaft
    der ABUS KG mit Sitz in Augsburg, die seit 1999 Sicherheitssysteme für kleine und mittlere Unternehmen sowie Privatpersonen
    anbietet. Das Produktportfolio umfasst unter anderem Alarmanlagen, Videoüberwachungs- und Zutrittskontrollsystemen.

    Das \acrlong*{portal} dient dazu, solche Geräte zu verwalten und zu warten. Dort kann man den Status eines
    Geräts und ggf. dessen Komponenten ansehen, Gerätenachrichten abonnieren und Logs einsehen sowie ein Gerät
    aus der Ferne konfigurieren. Es gibt vier Rollen, die das \acrshort{portal} in unterschiedlichem Umfang und
    für unterschiedliche Zwecke nutzen können: \Gls{endnutzer}, \Gls{facherrichter}, \Gls{ts-gls} und \Gls{pm-gls}.
    Jede Rolle ist im \hyperref[sec:glossary]{Glossar} detailliert beschrieben.

    Das conplement-Entwicklungsteam vom \acrshort{portal} besteht aus einer Projektmanagerin, einem
    Softwarearchitekten, zwei Fullstack-Entwicklern und einem Auszubildenden (mir), ebenfalls als
    Fullstack-Entwickler tätig. Das Team arbeitet eng mit dem Product Owner von \acrshort{abus} zusammen
    und setzt agile Methoden wie Kanban ein, um die Entwicklung sowie den Betrieb effizient zu gestalten.

    \subsection{Projektziel}

    Das Ziel des Projekts ist die Implementierung eines neuen Features im \acrshort{portal},
    das dem \acrshort{ts} ermöglicht, die Häufigkeit und Art von Gerätenachrichten für ein
    bestimmtes Gerät über einen bestimmten Zeitraum zu analysieren. Dies soll auch ohne
    Freigabe durch den \Gls{facherrichter} oder \Gls{endnutzer} möglich sein. Ein \acrshort{pm}
    soll ebenfalls Zugriff auf diese Funktionalität haben, aber die Hauptzielgruppe ist der
    \acrshort{ts}.

    Ein Auszug aus den Anforderungen des Kunden beschreibt das Ziel wie folgt (im Original auf Englisch):
    \begin{quote}
        \textit{As role technical support or pm, I want to see diagrams and overviews of system messages on the new
        ``System Analysis'' page so that I can identify possible issues and draw conclusions about potential errors.}

        Acceptance criteria:
        \begin{itemize}
            \item The page displays a graph/chart visualizing system messages over time (for the exact type of chart please make an appropriate proposal)
            \item The system messages can be filtered by time range, allowing me to view data from the past (e.g. last 24h, last 7 days, …)
            \item The diagram updates when a filter is applied
            \item The default filter is last 7 days
        \end{itemize}
    \end{quote}

    Außerdem wurde ein Mockup von dem UI-Designer von \acrshort{abus} erstellt, das als visuelle
    Referenz für die Frontend-Implementierung dient. Die Anforderungen sind bewusst etwas offen formuliert,
    um dem Entwicklungsteam die Freiheit zu geben, die beste technische Lösung zu finden.

    \subsection{Projektbegründung}

    Momentan kann ein \acrshort{ts} nur bei Geräten der Produktlinie \textit{Secoris}\footnote{
        Für andere Gerätetypen ist die Anzeige von Logs im \acrshort{portal} erst für die Zukunft geplant.
    }
    die Gerätenachrichten ansehen. Dazu muss er jedoch aus Datenschutzgründen eine Freigabe vom \Gls{facherrichter} oder
    \Gls{endnutzer} einholen, was den Support-Prozess verlangsamt. Außerdem hat die Secoris-Log-Tabelle keine grafische
    Darstellung der Nachrichtenhäufigkeit, was die Analyse und Erkennung von Mustern erschwert.

    Das neue Feature soll diese Probleme lösen und die Effizienz des technischen Supports verbessern. Die Einsicht bleibt nur
    auf die Daten beschränkt, die für die Störungsbehebung notwendig sind. Somit wird Datenschutz weiterhin gewährleistet.
    Durch die schnellere Problemlösung können sowohl monetäre als auch nicht-monetäre Vorteile für \acrshort{abus} entstehen,
    wie in der \ref{sec:profitability-analysis} beschrieben.

    \subsection{Projektschnittstellen}

    Die technischen Schnittstellen des Projekts sind wie folgt:
    \begin{itemize}
        \item Die Daten für die Analyse stammen aus der bestehenden \Gls{mongodb}-Datenbank. Sie müssen ausgelesen
        und aggregiert werden.
        \item Dies wird im bestehenden Backend passieren, das in C\# mit ASP.NET implementiert ist.
        \item Die Auswertungsansicht muss in das bestehende Frontend vom \acrshort{portal} integriert werden.
        \item Für die DevOps-Prozesse (Build, Tests, Deployment) werden bestehende Pipelines und Tools in
        \textit{Azure DevOps} verwendet.
    \end{itemize}

    Die personellen Schnittstellen des Projekts sind wie folgt:
    \begin{itemize}
        \item Abstimmung und Code-Reviews mit den beiden Fullstack-Entwicklern im Team.
        \item Bei Fragen oder Verbesserungsvorschlägen zu den Anforderungen und dem Design
        wird der Product Owner von \acrshort{abus} kontaktiert. Die Ergebnisse des Projekts
        werden ebenfalls ihm präsentiert.
        \item Für das UI-Design wird der UI-Designer von \acrshort{abus} einbezogen.
        \item Die Nutzer des neuen Features sind \acrshort{ts} und \acrshort{pm} von \acrshort{abus}.
    \end{itemize}

    \subsection{Projektabgrenzung}

    Die folgenden Punkte sind nicht Teil des Projekts:
    \begin{itemize}
        \item Datenbank-Änderungen: Sowohl die Datenbank-Struktur als auch die
        Daten selbst sind bereits vorhanden und müssen nicht geändert werden. Sie
        werden nur gelesen und ausgewertet.
        \item Das \acrshort{portal}: Das Projekt umfasst nur die Implementierung
        des neuen Features in die \textbf{bestehende} Anwendung.
        \item \Gls{rwd}: Die bestehende Anwendung ist nicht für mobile Geräte
        optimiert. Daher wird das neue Feature ebenfalls nur für die Desktop-Ansicht
        implementiert.
        \item Diagramme: Die Darstellung der Daten in Diagrammen wird mit einer
        externen Bibliothek umgesetzt
    \end{itemize}

    \subsection{Projektphasen}

    Das Projekt wird nach dem \textit{Wasserfallmodell} durchgeführt, da die Anforderungen
    zu Beginn des Projekts klar definiert sind und sich voraussichtlich nicht ändern werden.
    Die einzelnen Phasen des Projekts sind wie folgt:
    \begin{itemize}
        \item Analysephase
        \item Konzeptionsphase
        \item Implementierungsphase
        \item Abschlussphase
    \end{itemize}

    In der folgenden Tabelle ist eine vollständige Zeitplanung mit den jeweiligen Projektphasen
    dargestellt. Die Nummerierung der einzelnen Schritte entspricht der Nummerierung 
    in der Zeitplanung des Projektantrags. Die Planungsphase wurde jedoch in zwei separate
    Phasen aufgeteilt: Analysephase und Konzeptionsphase.

    \begin{table}[H]
        \centering
        \begin{tabular}{|l|p{0.3\linewidth}|l|r|}
            \hline
            \textbf{Nr.}\textsuperscript{*} & \textbf{Aufgabe}                                                   & \textbf{Phase}           & \textbf{Geplante Zeit (Stunden)} \\
            \hline
            1.1  & Wirtschaftlichkeitsanalyse                                                  & Analyse         & 2                       \\
            1.2  & Ist-Analyse des Gerätenachrichten-Features im Backend                       & Analyse         & 3                       \\
            1.3  & Ist-Analyse der Geräteansicht des technischen Supports im Frontend          & Analyse         & 1                       \\
            1.4  & Planung der technischen Implementierung, Abstimmung mit anderen Entwicklern & Konzeption      & 5                       \\
            1.5  & Entwurf eines REST-API-Endpunkts                                            & Konzeption      & 1                       \\
            2.1  & Frontend-Entwicklung                                                        & Implementierung & 18                      \\
            2.2  & Frontend-Tests                                                              & Implementierung & 5                       \\
            2.3  & Backend-Entwicklung                                                         & Implementierung & 18                      \\
            2.4  & Backend-Tests                                                               & Implementierung & 8                       \\
            2.5  & Manueller Systemtest lokal                                                  & Implementierung & 1                       \\
            3.1  & Code-Review mit einem anderen Entwickler                                    & Abschluss       & 2                       \\
            3.2  & Deployment auf Testumgebung                                                 & Abschluss       & 2                       \\
            3.3  & Präsentation der Ergebnisse dem Kunden                                      & Abschluss       & 1                       \\
            4.1  & Projektdokumentation                                                        & Abschluss       & 10                      \\
            4.2  & Entwicklerdokumentation                                                     & Abschluss       & 3                       \\
            \rowcolor{lightgreen} \textbf{} & \textbf{Gesamt}                                                        &                 & \textbf{80}             \\
            \hline
        \end{tabular}
        \caption{Zeitplanung des Projekts}
        \vspace{0.5em}
        \footnotesize\textsuperscript{*}laut Projektantrag
    \end{table}

    Ergebnisse einiger Phasen sind nicht in eigenen Abschnitten dieser Dokumentation dargestellt, sondern sind
    in den jeweiligen Phasen integriert (z.B. die Planung der technischen Implementierung umfasst sowohl Backend-
    als auch Frontend-Architektur und wird im Abschnitt \ref{sec:conceptualize} beschrieben).